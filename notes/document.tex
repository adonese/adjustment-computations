\documentclass[]{scrartcl}

%opening
\title{Assignment 2}
\author{Adjustment Computations}
\usepackage{amsmath}
\usepackage{booktabs}
\usepackage{siunitx}
\usepackage{hyperref}

\begin{document}

\maketitle

\section*{Notes About the assignment}
Generally your assignments would consist of two sections. You have practice problems that you need to solve by your hands--that will help you understand the materials even better. You also have programming problems, where you need to build a program that solves specific problem. Programming assignments are very important, in practice you often end up having observations that cannot be solved by hands.

Please, only submit your work! Do not cheat! I've developed a very buggy algorithm to compare your works.

\section{Problem 1}
Compute the most probable values (MPV) for X, Y using the method of least squares.
\begin{eqnarray}
\label{eqn:1}
X + 2Y &= 10.5\\
2X - 3Y &= 5.5\\
2X - Y &= 10.0
\end{eqnarray}

solve this problem using
\begin{itemize}
	\item The matrix method (with your calculator, not MATLAB)
	\item By solving the equations simultaneously (reduce the equations in \eqref{eqn:1} to 2, using Least squares)
\end{itemize}

\section{Problem 2}
This problem might be a little bit harder.\\
Using the method of least squares, compute the most probable values (MPV) for the three angles observed to close the horizon at station Red?
\\
The angles observed values, and their standard deviation (or, standard error) are as follows

\begin{table}[]
	\centering
	\label{table:table-1}
	\begin{tabular}{@{}lll@{}}
		\toprule
		id & observation [d,m,s]& STD [sec]\\
		\midrule
		1 & \ang{114; 23; 05} & $\pm 2.5$\\
		2 & \ang{138; 17; 59} & $\pm 1.5$\\
		3 & \ang{107; 19; 03} & $\pm 4.9$\\
		\bottomrule
	\end{tabular}
\end{table}

\paragraph{HINT.}Two hints, well this is a \textit{weighted} least squares problem! It is not very different from the standard LS solution, well the standard LS solution is as follows,

\begin{eqnarray}
E = (X^TX)^{-1}(X^TY)
\end{eqnarray}
The weighted solution is as follows,
\begin{eqnarray}
E = (X^TWX)^{-1}(X^TWY)
\end{eqnarray}
Where $E$ is the error vector (?), $W$ is the weight vector, $X$ is the coefficients vector. $Y$ corresponds to the constant term.
Again, the notations might be somewhat messy, but this basically because least squares are used everywhere. So, each community are using their own notations for least squares.
\paragraph{HINT 2.}To solve this problem you need to remember that for the sum of the angles should be equal to 360. The weights are computed using this equation
\begin{equation}
	w_i = \frac{1}{\sigma^2_i}
\end{equation}
where $\sigma^2_i$ is the $i^{th}$ standard deviation.
\\
You can solve this problem using any least squares method you want. BONUS points will be given for those who will solve it using several methods.

\section{Submission}
For your submission, would you please upload your work on this website \href{https://www.gradescope.com}{https://www.gradescope.com}. You need to use \textbf{9VYXEM} for the course code.
\end{document}
