% This is LLNCS.DEM the demonstration file of
% the LaTeX macro package from Springer-Verlag
% for Lecture Notes in Computer Science,
% version 2.4 for LaTeX2e as of 16. April 2010
%
\documentclass{llncs}
%
\usepackage{amsmath}
\usepackage{makeidx}  % allows for indexgeneration
%
\begin{document}
%
\frontmatter          % for the preliminaries
%
\pagestyle{headings}  % switches on printing of running heads
\addtocmark{Hamiltonian Mechanics} % additional mark in the TOC
%
\chapter*{Preface}
%
This textbook is intended for use by students of physics, physical
chemistry, and theoretical chemistry. The reader is presumed to have a
basic knowledge of atomic and quantum physics at the level provided, for
example, by the first few chapters in our book {\it The Physics of Atoms
and Quanta}. The student of physics will find here material which should
be included in the basic education of every physicist. This book should
furthermore allow students to acquire an appreciation of the breadth and
variety within the field of molecular physics and its future as a
fascinating area of research.

For the student of chemistry, the concepts introduced in this book will
provide a theoretical framework for that entire field of study. With the
help of these concepts, it is at least in principle possible to reduce
the enormous body of empirical chemical knowledge to a few basic
principles: those of quantum mechanics. In addition, modern physical
methods whose fundamentals are introduced here are becoming increasingly
important in chemistry and now represent indispensable tools for the
chemist. As examples, we might mention the structural analysis of
complex organic compounds, spectroscopic investigation of very rapid
reaction processes or, as a practical application, the remote detection
of pollutants in the air.

\vspace{1cm}
\begin{flushright}\noindent
April 1995\hfill Walter Olthoff\\
Program Chair\\
ECOOP'95
\end{flushright}
%

\chapter{Least Squares for nonlinear equations}

\section{Something}
Sorry for the last section. I was very tired, and I think that you might have missed some of the key points behind this very important topic: linearization. I decided to write down what I believe to be very important. I still will revise it tomorrow, or the next week though. 

In typical surveying works we often run it situations where we need to use the least squares to adjust some nonlinear functions. You often see that in triangulation and traverse networks. In geodetic netowrks you also need to do least squares in nonlinear set of equations. In this section we are trying to get more intuitions about the use of least squares in nonlinear equations. 
\begin{equation}
AX = L,
\end{equation}


which is our old least squares equation. To see exactly when the \textit{nonlinearty} issue arises, let us see this example.


\begin{align}
\label{eqn:1}
x + y - 2y^2 &= -4\\
x^2 + y^2 &= 8\\
3x^2 - y^2 &= 7.7.
\end{align}

and this example.

\begin{align}
\label{eqn:2}
A x_a^2 + B x_a + C &= y_a + \upsilon_a\\
A x_b^2 + B x_b + C &= y_b + \upsilon_b\\
A x_c^2 + B x_c + C &= y_c + \upsilon_c\\
A x_d^2 + B x_d + C &= y_d + \upsilon_d\\
A x_e^2 + B x_e + C &= y_e + \upsilon_e.
\end{align}

How can you say either of \eqref{eqn:1} or \eqref{eqn:2} is linear. It solely depends on what you are trying to solve for. 
\\
So, basically in \eqref{eqn:1}, we are trying to solve for $x$, and $y$, and it is clear that they are not linearly related.  You cannot trait as if it is just a linear equation. In \eqref{eqn:2} however, we want to solve for $A,B$ and $C$. So, \eqref{eqn:1}, is nonlinear equation, and equation \eqref{eqn:2} is a linear one.

\subsection{Taylor's series}



\clearpage
\addtocmark[2]{Author Index} % additional numbered TOC entry
\renewcommand{\indexname}{Author Index}
\printindex
\clearpage
\addtocmark[2]{Subject Index} % additional numbered TOC entry
\markboth{Subject Index}{Subject Index}
\renewcommand{\indexname}{Subject Index}
%                                                           clmomu01.ind
%-----------------------------------------------------------------------
% CLMoMu01 1.0: LaTeX style files for books
% Sample index file for User's guide
% (c) Springer-Verlag HD
%-----------------------------------------------------------------------
\begin{theindex}
\item Absorption\idxquad 327
\item Absorption of radiation \idxquad 289--292,\, 299,\,300
\item Actinides \idxquad 244
\item Aharonov-Bohm effect\idxquad 142--146
\item Angular momentum\idxquad 101--112
\subitem algebraic treatment\idxquad 391--396
\item Angular momentum addition\idxquad 185--193
\item Angular momentum commutation relations\idxquad 101
\item Angular momentum quantization\idxquad 9--10,\,104--106
\item Angular momentum states\idxquad 107,\,321,\,391--396
\item Antiquark\idxquad 83
\item $\alpha$-rays\idxquad 101--103
\item Atomic theory\idxquad 8--10,\,219--249,\,327
\item Average value\newline ({\it see also\/} Expectation value)
15--16,\,25,\,34,\,37,\,357
\indexspace
\item Baker-Hausdorff formula\idxquad 23
\item Balmer formula\idxquad 8
\item Balmer series\idxquad 125
\item Baryon\idxquad 220,\,224
\item Basis\idxquad 98
\item Basis system\idxquad 164,\,376
\item Bell inequality\idxquad 379--381,\,382
\item Bessel functions\idxquad 201,\,313,\,337
\subitem spherical\idxquad 304--306,\, 309,\, 313--314,\,322
\item Bound state\idxquad 73--74,\,78--79,\,116--118,\,202,\, 267,\,
273,\,306,\,348,\,351
\item Boundary conditions\idxquad 59,\, 70
\item Bra\idxquad 159
\item Breit-Wigner formula\idxquad 80,\,84,\,332
\item Brillouin-Wigner perturbation theory\idxquad 203
\indexspace
\item Cathode rays\idxquad 8
\item Causality\idxquad 357--359
\item Center-of-mass frame\idxquad 232,\,274,\,338
\item Central potential\idxquad 113--135,\,303--314
\item Centrifugal potential\idxquad 115--116,\,323
\item Characteristic function\idxquad 33
\item Clebsch-Gordan coefficients\idxquad 191--193
\item Cold emission\idxquad 88
\item Combination principle, Ritz's\idxquad 124
\item Commutation relations\idxquad 27,\,44,\,353,\,391
\item Commutator\idxquad 21--22,\,27,\,44,\,344
\item Compatibility of measurements\idxquad 99
\item Complete orthonormal set\idxquad 31,\,40,\,160,\,360
\item Complete orthonormal system, {\it see}\newline
Complete orthonormal set
\item Complete set of observables, {\it see\/} Complete
set of operators
\indexspace
\item Eigenfunction\idxquad 34,\,46,\,344--346
\subitem radial\idxquad 321
\subsubitem calculation\idxquad 322--324
\item EPR argument\idxquad 377--378
\item Exchange term\idxquad 228,\,231,\,237,\,241,\,268,\,272
\indexspace
\item $f$-sum rule\idxquad 302
\item Fermi energy\idxquad 223
\indexspace
\item H$^+_2$ molecule\idxquad 26
\item Half-life\idxquad 65
\item Holzwarth energies\idxquad 68
\end{theindex}

\end{document}
