\documentclass[]{scrartcl}

%opening
\title{Assignment 2}
\author{Adjustment Computations}
\usepackage{amsmath}
\usepackage{booktabs}
\usepackage{siunitx}
\usepackage{hyperref}
\usepackage{minted}

\begin{document}

\maketitle

\section*{Notes About the assignment}
Generally your assignments would consist of two sections. You have practice problems that you need to solve by your hands--that will help you understand the materials even better. You also have programming problems, where you need to build a program that solves specific problem. Programming assignments are very important, in practice you often end up having observations that cannot be solved by hands.

Please, only submit your work! Do not cheat! I've developed a very buggy algorithm to compare your works.
\\
Last note, you need to send me your submission via (gradescope, preferably, or to my email. I will not accept your handwritten submissions.)

\section{Practical Problems}
\paragraph{Problem 1.}
Using the conditional equations method, compute the most probable values for the three interior angels of a triangle measured as follows
\begin{table}[]
	\centering
	\label{table:table-1}
	\begin{tabular}{@{}lll@{}}
		\toprule
		\emph{Angle} & \emph{Value} [dms] & \emph{STD} [sec]\\
		\midrule
		1 & \ang{58; 14; 56} & $\pm 5.1$\\
		2 & \ang{67; 02; 34} & $\pm 4.3$\\
		3 & \ang{52; 42; 40} & $\pm 2.6$\\
		\bottomrule
	\end{tabular}
\end{table}

\paragraph{Problem 2. Not a least squares} During a construction project the difference in elevation from the deck to the surface of the underpass is observed \textit{four} times using differential leveling. The observations, and the lengths of the lines are as follows.

\begin{table}[]
	\centering
	\label{table:table-2}
	\begin{tabular}{@{}lll@{}}
		\toprule
		\emph{Route} & $\triangle Elev (m)$ & \emph{Length} [m]\\
		\midrule
		1 & 5.003 & 36.032\\
		2 & 4.978 & 52.305\\
		3 & 5.012 & 48.897\\
		4 & 4.995 & 38.902\\
		\bottomrule
	\end{tabular}
\end{table}

Compute:
\begin{enumerate}
	\item The weighted mean of the elevation difference.
	\item The standard deviation of the weighted mean.
	\item The standard deviation for each weighted observation.
\end{enumerate}

\section{Programming Problems}

\paragraph{Problem 3.}Write a function that converts from degrees, minutes, seconds to decimal degrees. Your function should work like the following

\begin{minted}{matlab}
function deg = dms2deg(a)
% dms2deg(a)
% dms2deg([12, 21 , 42])
% ans = 12.362
% Inputs:
% a: is the angle in a vector form. (n,1) e.g., 
% a = [12, 21, 42]
% Output:
% deg =  12.362
your code goes here...
end
\end{minted}

Remember that the equation for converting from dms to decimal degrees is the following
\begin{equation}
	decimal = degree + \frac{minutes}{60} + \frac{seconds}{3600}
\end{equation}
\paragraph{Problem 4.} Write a function that solves problem 2.

\section{Submission}
For your submission, would you please upload your work on this website \href{https://www.gradescope.com}{https://www.gradescope.com}. You need to use \textbf{9VYXEM} for the course code.
\end{document}
